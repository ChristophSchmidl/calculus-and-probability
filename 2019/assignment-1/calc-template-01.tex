%
% Although we try to provide a template that completely
% matches the corresponding assignment, we do expect you
% to check that you have indeed answered all questions.
%

% ALSO VERY IMPORTANT:
% This is just a template to help you with the LaTeX part of the assignment.
% So you may change it completely according to your own wishes!
%

\documentclass[a4paper]{article}
% Typically the 'article' class is appropriate for assignments.
% And we print it on a4, so we include that as well.

\usepackage{a4wide}
% To decrease the margins and allow more text on a page.

\usepackage{graphicx}
% To deal with including pictures.

\usepackage{enumerate}
% To provide a little bit more functionality than with LaTeX's default
% enumerate environment.

\usepackage{array} 
% To provide a little bit more functionality than with LaTeX's default
% array environment.

\usepackage[american]{babel}
% Use this if you want to write the document in US English. It takes care of
% (usually) proper hyphenation.
% If you want to write your answers in Dutch, please replace 'american'
% by 'dutch'.
% Note that after a change it may be that the first compilation of LaTeX
% fails. That is normal and caused by the fact that in auxiliary files
% from previous runs, there may still be a \selectlanguage{american}
% around, which is invalid if 'american' is not incorporated with babel.

\usepackage{amssymb}
% This package loads mathematical things like the fonts for the blackboard
% bold for the set of natural numbers.

\usepackage{amsmath}
\usepackage{clipboard}

\newcommand{\exercise}[2]{\subsection*{Exercise #1}{#2}}
\newcommand{\exerciseenum}[2]{\subsection*{Exercise #1}{\begin{enumerate}[a)]#2\end{enumerate}}}
% We defined our own commands to make it easy to present all the
% exercises in the same style. The first one does not automatically
% start an 'enumerate' list, the second one does.
% The [2] means that our command needs two arguments.
% The #1 and the #2 indicate where we use these arguments in the
% command.
% There are several ways to have automatic numbering for the exercises,
% but here we have chosen to use a subsection for this and use manual
% numbering. This is because maybe not everyone will be able to do hand in
% all exercises.
% Note that we add the '*' to make sure that the subsection is not numbered.
% (Since we don't have a \section, the numbers for a subsection would be
% ugly like 0.1, 0.2 et cetera.
% The environment 'enumerate' automatically numbers the items in this list.
% The optional [a)] makes sure that the list will be like a), b), c) et cetera.


\newcommand{\set}[1]{\ensuremath{\left\{{#1}\right\}}}
% This command puts curly braces around its argument, so it becomes
% a set. The \left and \right make sure that the braces grow in size
% if the contents of the set are large symbols.

\newcommand{\setbuild}[2]{\ensuremath{\set{{#1}\mid{#2}}}}
% We also introduce a shortcut for using the set builder notation.
% Do you understand what it does?

% And the next series of commands gives you some of the default sets
% that were in the slides.
\newcommand{\TT}{\ensuremath{\mathbb{T}}}
\newcommand{\FF}{\ensuremath{\mathbb{F}}}
\newcommand{\NN}{\ensuremath{\mathbb{N}}}
\newcommand{\NNp}{\ensuremath{\mathbb{N}^{+}}}
\newcommand{\ZZ}{\ensuremath{\mathbb{Z}}}
\newcommand{\ZZp}{\ensuremath{\mathbb{Z}^{+}}}
\newcommand{\QQ}{\ensuremath{\mathbb{Q}}}
\newcommand{\QQp}{\ensuremath{\mathbb{Q}^{+}}}
\newcommand{\RR}{\ensuremath{\mathbb{R}}}
\newcommand{\RRp}{\ensuremath{\mathbb{R}^{+}}}
\newcommand{\CC}{\ensuremath{\mathbb{C}}}

% An the next command gives a shorthand for the power set of a given set.
\newcommand{\power}[1]{\ensuremath{{\cal P}\left({#1}\right)}}

\newcommand{\sol}[1]{\underline{\underline{#1}}}


\title{Calculus and Probability\\Assignment 1}

% Replace the placeholders by your real name, student number and 
% group (for the exercise hours)
\author{
 Christoph Schmidl  % Fill in your name
 \\ 
 s4226887  % Fill in your student number
 \\
 Group: not assigned yet  % Fill in your assigned group 
 \\
 Name of your TA : not assigned yet % Fill in the name of your student assistant
}

% In LaTeX everything before \begin{document} is called pre-amble.
% This is where you put all important settings. The real document
% starts after \begin{document}.
\begin{document}
\maketitle 
% \maketitle makes sure that the title is shown on the first page of
% the document.


% Now we use the command we defined earlier and give it the proper two
% parameters.
% Because the second parameter is long, we put a % directly after the
% opening curly brace {. This is not needed but makes the source file
% look a bit better.

%\newpage

% #####################################################################################
%
% Exercises are below. 
%
% #####################################################################################

\exerciseenum{6}{% 
% The function is zero for 3 distinct values
\item%a
$f(x) = x - x^3 = x(1-x^2)= x(1-x)(1+x)$\\
Therefore, $f(x) = 0$ if $x \in \{ -1,0,1 \}$



\item%b
% You can identify 4 different cases (regions) given the three values you found in item a
$(1-x^2) > 0 \Leftrightarrow |x| < 1 \Leftrightarrow x \in (-1,1)$ So we have to distinguish between four cases:

\begin{itemize}
	\item Case 1: $x < -1$. Then $x < 0$ and $(1-x^2) < 0$. Therefore, $f(x) > 0$
	\item Case 2: $x \in (-1,0)$. Then $x < 0$ and $(1-x^2) > 0$. Therefore, $f(x) < 0$
	\item Case 3: $x \in (0,1)$. Then $x > 0$ and $(1-x^2) > 0$. Therefore, $f(x) > 0$
	\item Case 4: $x > 1$.Then $x > 0$ and $(1-x^2) < 0$. Therefore, $f(x) < 0$
\end{itemize}


}

\exercise{7}{%
% You need to distinguis 3 cases depending on the relation between a and b
\begin{itemize}
	\item Case 1: When $a = b$, we get $y = \{ a \}$
	\item Case 2: When $b > a$ then $(b-a) \geq 0$. $y$ runs through all values in $(a,b)$
	\item Case 3: When $b < a$ then $(b-a) < 0$. $y$ runs through all values in $(b,a)$
\end{itemize}
}

\exerciseenum{8}{%
% Show that one of the following holds:
%   f(-x) = ... = f(x)  (even), or
%   f(-x) = ... = -f(x) (odd)
\item%a
$f(x) = 3x - x^3$

If we fill in $-x$, we get:\\
$f(-x) = 3(-x) - (-x)^3 = -3x + x^3 = -(3x - x^3) = -f(x)$.\\
Therefore, the function is odd. 
\item%b
$f(x) = \sqrt[3]{(1-x)^2} + \sqrt[3]{(1+x)^2}$ 

If we fill in $-x$, we get:\\
$f(-x) = \sqrt[3]{(1- (-x))^2} + \sqrt[3]{(1+(-x))^2} = \sqrt[3]{(1+x)^2} + \sqrt[3]{(1-x)^2} = f(x)$\\
Therefore, the function is even. 
}

\exerciseenum{9}{%
\item%a
$f(x) = \sqrt{7-x^2} + 1$\\
To be able to compute the square root, the following property has to hold: $7 - x^2 \geq 0$. So $D(f) = [-\sqrt{7}, \sqrt{7}]$ and $R(f) = [1, \sqrt{7} + 1]$
\item%b
$f(x) = \frac{1}{|x|}$\\
The only value which is excluded is zero because you cannot divide by it. Therefore, $D(f) = \mathbb{R} \backslash \{ 0\}$ and $R(f) = (0, \infty)$
}

\exerciseenum{10}{%
\item%a
$y(cx + d) = ax + b$ $\rightarrow$ $x(cy - a) = -dy + b$, so that $x = \frac{-dy + b}{cy - a}$. Therefore the inverse function $g$ is given by $g(x) = \frac{-dx + b}{cx - a}$
\item%b
It is equal to the original function when $d = -a$
}

\exerciseenum{11}{%
%In both case you need to simplify the expression first
\item%a
$\lim\limits_{x\to 2} \frac{x-2}{x^2+x-6}$ = 
$\lim\limits_{x\to 2} \frac{x-2}{(x-2)(x+3)}$ =
$\lim\limits_{x\to 2} \frac{1}{x+3}$

Therefore, $\lim\limits_{x\to 2} \frac{x-2}{x^2+x-6} = \frac{1}{5}$    
  
\item%b  
$\lim\limits_{x\to 1} \frac{x^2-4x+3}{x^2+x-2}$ =
$\lim\limits_{x\to 1} \frac{(x-1)(x-3)}{(x+2)(x-1)}$ = 
$\lim\limits_{x\to 1} \frac{(x-3)}{(x+2)}$

Therefore, $\lim\limits_{x\to 1} \frac{x^2-4x+3}{x^2+x-2} = -\frac{2}{3}$
}


\exercise{12}{%
% This was not discussed yet at the first lecture, but
% interpret "continuous" as being able to draw the graph without removing your pen from the paper, and
% you need to fit a straight line between two points.

First we need to find the inverse $f^{-1}(x)$:

\begin{align*}
x &= \frac{y}{2y + 3}\\
x(2y + 3) &= y \\
2xy + 3x &= y\\
2xy &= y - 3x\\
2xy - y &= -3x\\
y(2x - 2) &= -3x\\
y &= - \frac{3x}{2x-1}
\end{align*}

Hence, the inverse $f^{-1}(x) = - \frac{3x}{2x - 1}$, and therefore the endpoints we seek are $p_1 = f^{-1}(-2) = - \frac{6}{5} = -1.2$ and $p_2=f^{-1}(3) = - \frac{9}{5} = -1.8$. We then need to compute the line $y=m\cdot x + b$ of the line going through points $p_1$ and $p_2$ 
where $m$ is the slope of the function and $b$ the intercept.\\

Finding the slope:

\begin{align*}
m &= \frac{y_2 - y_1}{x_2 - x_1}\\
&= \frac{3 + 2}{-1.8 + 1.2}\\
&\approx -8.3
\end{align*}

}



\end{document}

