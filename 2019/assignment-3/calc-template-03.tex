%
% Although we try to provide a template that completely
% matches the corresponding assignment, we do expect you
% to check that you have indeed answered all questions.
%

% ALSO VERY IMPORTANT:
% This is just a template to help you with the LaTeX part of the assignment.
% So you may change it completely according to your own wishes!
%

\documentclass[a4paper]{article}
% Typically the 'article' class is appropriate for assignments.
% And we print it on a4, so we include that as well.

\usepackage{a4wide}
% To decrease the margins and allow more text on a page.

\usepackage{graphicx}
% To deal with including pictures.

\usepackage{enumerate}
% To provide a little bit more functionality than with LaTeX's default
% enumerate environment.

\usepackage{array} 
% To provide a little bit more functionality than with LaTeX's default
% array environment.

\usepackage[american]{babel}
% Use this if you want to write the document in US English. It takes care of
% (usually) proper hyphenation.
% If you want to write your answers in Dutch, please replace 'american'
% by 'dutch'.
% Note that after a change it may be that the first compilation of LaTeX
% fails. That is normal and caused by the fact that in auxiliary files
% from previous runs, there may still be a \selectlanguage{american}
% around, which is invalid if 'american' is not incorporated with babel.

\usepackage{amssymb}
% This package loads mathematical things like the fonts for the blackboard
% bold for the set of natural numbers.

\usepackage{amsmath}
\usepackage{clipboard}

\newcommand{\exercise}[2]{\subsection*{Exercise #1}{#2}}
\newcommand{\exerciseenum}[2]{\subsection*{Exercise #1}{\begin{enumerate}[a)]#2\end{enumerate}}}
% We defined our own commands to make it easy to present all the
% exercises in the same style. The first one does not automatically
% start an 'enumerate' list, the second one does.
% The [2] means that our command needs two arguments.
% The #1 and the #2 indicate where we use these arguments in the
% command.
% There are several ways to have automatic numbering for the exercises,
% but here we have chosen to use a subsection for this and use manual
% numbering. This is because maybe not everyone will be able to do hand in
% all exercises.
% Note that we add the '*' to make sure that the subsection is not numbered.
% (Since we don't have a \section, the numbers for a subsection would be
% ugly like 0.1, 0.2 et cetera.
% The environment 'enumerate' automatically numbers the items in this list.
% The optional [a)] makes sure that the list will be like a), b), c) et cetera.


\newcommand{\set}[1]{\ensuremath{\left\{{#1}\right\}}}
% This command puts curly braces around its argument, so it becomes
% a set. The \left and \right make sure that the braces grow in size
% if the contents of the set are large symbols.

\newcommand{\setbuild}[2]{\ensuremath{\set{{#1}\mid{#2}}}}
% We also introduce a shortcut for using the set builder notation.
% Do you understand what it does?

% And the next series of commands gives you some of the default sets
% that were in the slides.
\newcommand{\TT}{\ensuremath{\mathbb{T}}}
\newcommand{\FF}{\ensuremath{\mathbb{F}}}
\newcommand{\NN}{\ensuremath{\mathbb{N}}}
\newcommand{\NNp}{\ensuremath{\mathbb{N}^{+}}}
\newcommand{\ZZ}{\ensuremath{\mathbb{Z}}}
\newcommand{\ZZp}{\ensuremath{\mathbb{Z}^{+}}}
\newcommand{\QQ}{\ensuremath{\mathbb{Q}}}
\newcommand{\QQp}{\ensuremath{\mathbb{Q}^{+}}}
\newcommand{\RR}{\ensuremath{\mathbb{R}}}
\newcommand{\RRp}{\ensuremath{\mathbb{R}^{+}}}
\newcommand{\CC}{\ensuremath{\mathbb{C}}}

% An the next command gives a shorthand for the power set of a given set.
\newcommand{\power}[1]{\ensuremath{{\cal P}\left({#1}\right)}}

\newcommand{\sol}[1]{\underline{\underline{#1}}}


\title{Calculus and Probability\\Assignment 3}

% Replace the placeholders by your real name, student number and 
% group (for the exercise hours)
\author{
\global\Copy{name}{
 Christoph Schmidl  % Fill in your name
} \\ 
\global\Copy{snumber}{
 s4226887  % Fill in your student number
} \\
\global\Copy{group}{
 Group 3 - Gijs Hendriksen  % Fill in your assigned group
}}

% In LaTeX everything before \begin{document} is called pre-amble.
% This is where you put all important settings. The real document
% starts after \begin{document}.
\begin{document}
\maketitle 
% \maketitle makes sure that the title is shown on the first page of
% the document.


% Now we use the command we defined earlier and give it the proper two
% parameters.
% Because the second parameter is long, we put a % directly after the
% opening curly brace {. This is not needed but makes the source file
% look a bit better.

%\newpage

% #####################################################################################
%
% Exercises are below. They consist of a calculation part and a solution part. The solution
% part is automatically copied to the last page for correcting by the student asisstents.
% This is somewhat experimental.... If it doesn't work, copy-paste by hand. 
%
% #####################################################################################

\exerciseenum{6}{% 
\item%a
\begin{align*}
f(x) = \arccos(\cos(x^2)) = x^2
\end{align*}

Because arccos is the inverse of cos, we can rewrite the function to just $f(x) = x^2$. This makes finding the derivative rather simple and we get $f'(x) = 2x$.\\

The other option is to apply the chain rule:

\begin{align*}
f(x) &= \arccos(\cos(x^2))\\
f'(x) &= - \frac{1}{\sqrt{1 - \cos^2(x^2)}} * (-2x \sin(x^2))\\
&= \frac{2x \sin(x^2)}{\sqrt{1 - \cos^2(x^2)}} \; \; \; \text{Remember: } \sin^2(\theta) + \cos^2(\theta) = 1\\
&= \frac{2x \sin(x^2)}{ \sqrt{\sin^2(x^2)}}\\
&= \frac{2x \sin(x^2)}{\sin(x^2)}\\
&= 2x
\end{align*}

\global\Copy{exsix}{
  $f'(x) = 2x$
}

}

\exerciseenum{7}{% 
\item%a

We do not need L'Hopital's rule here.
\begin{align*}
\lim_{x \to \infty} \frac{x^2}{1 + e^{-x}} = \frac{\infty}{1} = \infty
\end{align*}
\global\Copy{exsevena}{
  $\lim_{x \to \infty} \frac{x^2}{1 + e^{-x}} = \frac{\infty}{1} = \infty$
}

\item%b
If we take the limit we get the undefined limit definition of $\frac{0}{0}$, therefore we can apply L'Hopital's rule. 
\global\Copy{exsevenb}{
  Answer 7b% Put here the final answer, which is then also copied to the form on the last page
}

}

\exerciseenum{8}{% 
\item%a
\begin{align*}
g(x) &= \cos(3x)\\
g'(x) &= 3 * (-\sin(3x))\\
g''(x) &= 9 * ( -\cos(3x))\\
g'''(x) &= 24 * \sin(3x)\\
g^{(2015)} &= 3^{2015} * \sin(3x)
\end{align*}
\global\Copy{exeight}{
  $g^{(2015)} = 3^{2015} * \sin(3x)$
}


}

\exerciseenum{9}{% 
\item%a
We can see the rools of $f$ right from the function definition, namely:

\begin{align*}
x_1 = -1\\
x_2 = 3
\end{align*}

We can get the y-intercept by putting in zero: $f(0) = -3$

\global\Copy{exninea}{
  Roots: $x_1 = -1$ and $x_2 = 3$. Y-intercept: $f(0) = -3$
}

\item%b
\begin{align*}
\lim_{x \to - \infty} f(x) = -\infty\\
\lim_{x \to + \infty} f(x) = +\infty
\end{align*}
\global\Copy{exnineb}{
  $\lim_{x \to - \infty} f(x) = -\infty$ and $\lim_{x \to + \infty} f(x) = +\infty$
}

\item%c
Using the product rule to find the first derivative:

\begin{align*}
f'(x) &= ((x^2 + 2x + 1)(x-3))'\\ 
&= (2x+2)(x-3)+x^2 + 2x + 1\\
&= 3x^2 - 2x -5
\end{align*}

The second derivative is therefore:

\begin{align*}
f''(x) = 6x - 2
\end{align*}

To find the zeros of $f'(x)$ when can rewrite the term as $(x-1)(3x - 5)$ which gives $x_1 = 1$ and $x_2 = \frac{5}{3}$. The only zero of $f''(x)$ is $x_1 = \frac{1}{3}$. We therefore found the x-coordinates for the critical points and insert them into the original function to get the actual points of the minima and maxima:

\begin{align*}
	f(-1) = 0\\
	f(\frac{5}{3}) = -\frac{2^8}{3^3} \approx 9.5
\end{align*}

This gets us the points: $(-1, 0)$ and $(\frac{5}{3}, 9.5)$\\

\begin{align*}
f''(-1) = - 14 < 0 \; \; \; \text{maximum}\\
f''(\frac{5}{3}) = 13 > 0 \; \; \; \text{minimum}
\end{align*} 

\global\Copy{exninec}{
  Maximum: $(-1,0)$, Maximum: $(\frac{5}{3}, 9.5)$
}

\item%d
The function is concave when: $f''(x) < 0$. Therefore, when $6x - 2 < 0$ which is the case when $x < \frac{1}{3}$
The function is convex when: $f''(x) > 0$. Therefore, when $6x - 2 > 0$ which is the case when $x > \frac{1}{3}$
Function $f$ has a point of inflection at $x = \frac{1}{3}$
\global\Copy{exnined}{
  The function is concave when: $f''(x) < 0$. Therefore, when $6x - 2 < 0$ which is the case when $x < \frac{1}{3}$
The function is convex when: $f''(x) > 0$. Therefore, when $6x - 2 > 0$ which is the case when $x > \frac{1}{3}$
Function $f$ has a point of inflection at $x = \frac{1}{3}$
}

}

\exerciseenum{10}{% 
\item%a
\begin{align*}
h(x) = \sin(x) - \frac{1}{3} \sin^3(x)\\
h'(x) = \cos(x) - \sin^2(x) \cos(x)\\
h'(x) = \cos(x) - (1 -  \cos^2(x)) \cos(x)\\
h'(x) = \cos(x) - (\cos(x) - \cos^3(x))\\
h'(x) = \cos(x) - \cos(x) + \cos^3(x)\\
h'(x) = \cos^3(x)
\end{align*}
\global\Copy{extena}{
  $h(x) = \sin(x) - \frac{1}{3} \sin^3(x)$
}

\item%b
\begin{align*}
f_1(x) = \frac{1}{2} \sin^2(x) + 1\\
f_2(x) = \frac{1}{2} \sin^2(x) + 2\\
f_3(x) = \frac{1}{2} \sin^2(x) + 3\\
f_i'(x) = \sin(x) \cos(x)
\end{align*}
\global\Copy{extenb}{
  $f_1(x) = \frac{1}{2} \sin^2(x) + 1, f_2(x) = \frac{1}{2} \sin^2(x) + 2,
f_3(x) = \frac{1}{2} \sin^2(x) + 3$
}

}


% If everything works correctly the page below is filled in automatically. If not, do some 
% copy-paste by hand. 

\newpage
\section*{Answer Form Assignment 3}

\noindent
\large
\begin{tabular}{|p{4cm}|p{11.5cm}|}
\hline
\textbf{Name} & \Paste{name} \\ \hline
\textbf{Student Number} & \Paste{snumber} \\ \hline
\end{tabular}

\vspace{0.5cm}
\noindent
\large
%\begin{tabular}{|m{2.5cm}|m{13cm}|}
\begin{tabular}{|m{1cm}@{}m{1.5cm}|p{13cm}|}
\hline
  \textbf{Question} && \textbf{Answer} \\
\hline
6 & (1pt) & \Paste{exsix} \\  \hline
7a  & (1pt) &  \Paste{exsevena} \\ \hline
7b  & (1pt) &  \Paste{exsevenb} \\ \hline
8a & (1pt) & \Paste{exeight} \\ \hline
9a & (1pt) &  \Paste{exninea} \\ \hline 
9b & (1pt) &  \Paste{exnineb} \\ \hline
9c & (1pt) &  \Paste{exninec} \\ \hline 
9d & (1pt) &  \Paste{exnined} \\ \hline
10a& (1pt) &  \Paste{extena} \\ \hline
10b& (1pt) &   \Paste{extenb} \\ \hline
\end{tabular}

\end{document}

