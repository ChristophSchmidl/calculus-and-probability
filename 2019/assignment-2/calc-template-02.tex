%
% Although we try to provide a template that completely
% matches the corresponding assignment, we do expect you
% to check that you have indeed answered all questions.
%

% ALSO VERY IMPORTANT:
% This is just a template to help you with the LaTeX part of the assignment.
% So you may change it completely according to your own wishes!
%

\documentclass[a4paper]{article}
% Typically the 'article' class is appropriate for assignments.
% And we print it on a4, so we include that as well.

\usepackage{a4wide}
% To decrease the margins and allow more text on a page.

\usepackage{graphicx}
% To deal with including pictures.

\usepackage{enumerate}
% To provide a little bit more functionality than with LaTeX's default
% enumerate environment.

\usepackage{array} 
% To provide a little bit more functionality than with LaTeX's default
% array environment.

\usepackage[american]{babel}
% Use this if you want to write the document in US English. It takes care of
% (usually) proper hyphenation.
% If you want to write your answers in Dutch, please replace 'american'
% by 'dutch'.
% Note that after a change it may be that the first compilation of LaTeX
% fails. That is normal and caused by the fact that in auxiliary files
% from previous runs, there may still be a \selectlanguage{american}
% around, which is invalid if 'american' is not incorporated with babel.

\usepackage{amssymb}
% This package loads mathematical things like the fonts for the blackboard
% bold for the set of natural numbers.

\usepackage{amsmath}
\usepackage{clipboard}

\newcommand{\exercise}[2]{\subsection*{Exercise #1}{#2}}
\newcommand{\exerciseenum}[2]{\subsection*{Exercise #1}{\begin{enumerate}[a)]#2\end{enumerate}}}
% We defined our own commands to make it easy to present all the
% exercises in the same style. The first one does not automatically
% start an 'enumerate' list, the second one does.
% The [2] means that our command needs two arguments.
% The #1 and the #2 indicate where we use these arguments in the
% command.
% There are several ways to have automatic numbering for the exercises,
% but here we have chosen to use a subsection for this and use manual
% numbering. This is because maybe not everyone will be able to do hand in
% all exercises.
% Note that we add the '*' to make sure that the subsection is not numbered.
% (Since we don't have a \section, the numbers for a subsection would be
% ugly like 0.1, 0.2 et cetera.
% The environment 'enumerate' automatically numbers the items in this list.
% The optional [a)] makes sure that the list will be like a), b), c) et cetera.


\newcommand{\set}[1]{\ensuremath{\left\{{#1}\right\}}}
% This command puts curly braces around its argument, so it becomes
% a set. The \left and \right make sure that the braces grow in size
% if the contents of the set are large symbols.

\newcommand{\setbuild}[2]{\ensuremath{\set{{#1}\mid{#2}}}}
% We also introduce a shortcut for using the set builder notation.
% Do you understand what it does?

% And the next series of commands gives you some of the default sets
% that were in the slides.
\newcommand{\TT}{\ensuremath{\mathbb{T}}}
\newcommand{\FF}{\ensuremath{\mathbb{F}}}
\newcommand{\NN}{\ensuremath{\mathbb{N}}}
\newcommand{\NNp}{\ensuremath{\mathbb{N}^{+}}}
\newcommand{\ZZ}{\ensuremath{\mathbb{Z}}}
\newcommand{\ZZp}{\ensuremath{\mathbb{Z}^{+}}}
\newcommand{\QQ}{\ensuremath{\mathbb{Q}}}
\newcommand{\QQp}{\ensuremath{\mathbb{Q}^{+}}}
\newcommand{\RR}{\ensuremath{\mathbb{R}}}
\newcommand{\RRp}{\ensuremath{\mathbb{R}^{+}}}
\newcommand{\CC}{\ensuremath{\mathbb{C}}}

% An the next command gives a shorthand for the power set of a given set.
\newcommand{\power}[1]{\ensuremath{{\cal P}\left({#1}\right)}}

\newcommand{\sol}[1]{\underline{\underline{#1}}}


\title{Calculus and Probability\\Assignment 2}

% Replace the placeholders by your real name, student number and 
% group (for the exercise hours)
\author{
\global\Copy{name}{
 Christoph Schmidl  % Fill in your name
} \\ 
\global\Copy{snumber}{
 s4226887  % Fill in your student number
} \\
\global\Copy{group}{
 Group: not assigned? Where can I find this?  % Fill in your assigned group
}}

% In LaTeX everything before \begin{document} is called pre-amble.
% This is where you put all important settings. The real document
% starts after \begin{document}.
\begin{document}
\maketitle 
% \maketitle makes sure that the title is shown on the first page of
% the document.


% Now we use the command we defined earlier and give it the proper two
% parameters.
% Because the second parameter is long, we put a % directly after the
% opening curly brace {. This is not needed but makes the source file
% look a bit better.

%\newpage

% #####################################################################################
%
% Exercises are below. They consist of a calculation part and a solution part. The solution
% part is automatically copied to the last page for correcting by the student asisstents.
% This is somewhat experimental.... If it doesn't work, copy-paste by hand. 
%
% #####################################################################################


\exerciseenum{7}{% 
\item%a

Dividing by $x^3$ makes the limit more obvious:

\begin{align*}
\lim_{x \to -\infty} \frac{x^3 + 2x^2 + 2}{3x^3 + x + 4} = \lim_{x \to -\infty} \frac{1 + \frac{2}{x} + \frac{2}{x^3}}{3 + \frac{1}{x^2} + 4 \frac{4}{x^3}} = \frac{1}{3}
\end{align*}

The numerator approaches 1 and the denominator 3.

\global\Copy{exsevena}{
  $\lim_{x \to -\infty} \frac{x^3 + 2x^2 + 2}{3x^3 + x + 4} = \frac{1}{3}$
}

\item%b

Dividing by $x^2$ makes the limit more obvious:

\begin{align*}
\lim_{x \to \infty} \frac{2x + 1}{ x^2 + x} = \lim_{x \to \infty} \frac{\frac{2}{x} + \frac{1}{x^2}}{ 1 + \frac{1}{x}} = 0
\end{align*}

The numerator approaches 0 and the denominator 1.

\global\Copy{exsevenb}{
  $\lim_{x \to \infty} \frac{2x + 1}{ x^2 + x} = 0$
}

}

\exerciseenum{8}{% 
\item

Derivative of the function $f(x) = 2x + 3$ for any point $a$ in $\mathbb{R}$:\\

\begin{align*}
f'(a) &= \lim_{h \to 0} \frac{f(a + h) - f(a)}{h}\\
f'(a) &= \lim_{h \to 0} \frac{2(a + h) + 3 - (2a + 3)}{h}\\
&= \lim_{h \to 0} \frac{2a + 2h + 3 - 2a - 3}{h}\\
&= \lim_{h \to 0} \frac{2h}{h}\\
&= 2
\end{align*}


\global\Copy{exeight}{
  $f'(a) = \lim_{h \to 0} \frac{f(a + h) - f(a)}{h} = \lim_{h \to 0} \frac{2h}{h} = 2$, 
  $f'(a) = 2$ for every $a$ in $\mathbb{R}$
}

}

\exerciseenum{9}{% 
\item%a

Applying the chain rule:

\begin{align*}
f'(x) = - \frac{1}{ (1 + \frac{1}{x})^2} * (- \frac{1}{x^2} ) = \frac{1}{x^2 (1 + \frac{1}{x})^2}
\end{align*}

Therefore we already got our slope $m$ in the known formula: y = mx + b\\
The tangent line at x = a is therefore:
\begin{align*}
y = \frac{1}{a^2(1 + \frac{1}{a})^2}x + b
\end{align*}

\global\Copy{exninea}{
  Answer 9a% Put here the final answer, which is then also copied to the form on the last page
}

\item%b
\dots  % Put here your calculation
\global\Copy{exnineb}{
  Answer 9b% Put here the final answer, which is then also copied to the form on the last page
}

}

\exerciseenum{10}{% 
\item%a
We already know that $(e^x)' = e^x$ and $(tan(x))' = \frac{1}{\cos^2(x)}$. By using the chain rule we get:

\begin{align*}
f(x) &= exp(tan(x))\\
f'(x) &= exp(\tan(x)) \frac{1}{\cos^2(x)} = \frac{exp(\tan(x))}{\cos^2(x)}
\end{align*}
\global\Copy{extena}{
  $f'(x) = exp(\tan(x)) \frac{1}{\cos^2(x)} = \frac{exp(\tan(x))}{\cos^2(x)}$
}

\item%b
We already know that $(\ln(x))' = \frac{1}{x}$ and $(\cos(x))' = - \sin(x)$. By using the chain rule we get:

\begin{align*}
f(x) &= - \ln(\cos(x))\\
f'(x) &= - \frac{1}{\cos(x)} * (- \sin(x)) = \frac{\sin(x)}{\cos(x)} = \tan(x)
\end{align*}

\global\Copy{extenb}{
  $f'(x) = - \frac{1}{\cos(x)} * (- \sin(x)) = \frac{\sin(x)}{\cos(x)} = \tan(x)$
}

}

\exerciseenum{11}{% 
\item%a

\begin{align*}
f(x) = (exp(x))^{exp(x)}
\end{align*}

\begin{itemize}
	\item Logarithmic differentation: $\ln( (exp(x)^{exp(x)} )) = x * exp(x)$. $\frac{f'(x)}{f(x)} = exp(x)(x + 1)$. $f'(x) = (exp(x)^{exp(x+1)}) (x+1) = exp(x^{(exp(x) +1)}) (x+1)$
	\item Chain rule: $g(x) = x^x$ and $h(x) = exp(x)$. $f(x) = (g \circ h)$. $g'(x) = x^x (log(x+1))$ and $h'(x) = exp(x)$. $f'(x) = g'(h(x)) h'(x) = (exp(x))^{exp(x)(x+1)} exp(x) = exp(x^{(exp(x) +1)}) (x+1)$
\end{itemize}

\global\Copy{exelevena}{
  $f'(x) = (exp(x)^{exp(x+1)}) (x+1) = exp(x^{(exp(x) +1)}) (x+1)$
}

\item%b
\begin{align*}
f(x) = \sqrt{x - 2}, \; \; \text{compute} (f^{-1})'(x) \; \; ( \text{for} \; x > 2 )
\end{align*}
\begin{itemize}
	\item Computing the inverse and differentiate: $f^{-1}(x) = x^2 + 2$. $(f^{-1})'(x) = 2x$
	\item Inverse rule: $f^{-1}(x) = x^2 + 2$. $f'(x) = \frac{1}{2 \sqrt{x - 2}}$. Therefore: $(f^{-1})'(x) = \frac{1}{f'(f^{-1}(x))} = \frac{1}{\frac{1}{2 \sqrt{(x^2 + 2) - 2}}} = \frac{1}{\frac{1}{2x}} = 2x$
\end{itemize}


\global\Copy{exelevenb}{
 $f^{-1}(x) = x^2 + 2$. $(f^{-1})'(x) = 2x$
}

}

\exerciseenum{12}{% 
\item
text
\global\Copy{exetwelve}{
  
}

}


% If everything works correctly the page below is filled in automatically. If not, do some 
% copy-paste by hand. 

\newpage
\section*{Answer Form Assignment 2}

\noindent
\large
\begin{tabular}{|p{4cm}|p{11.5cm}|}
\hline
\textbf{Name} & \Paste{name} \\ \hline
\textbf{Student Number} & \Paste{snumber} \\ \hline
\end{tabular}

\vspace{0.5cm}
\noindent
\large
%\begin{tabular}{|m{2.5cm}|m{13cm}|}
\begin{tabular}{|m{1cm}@{}m{1.5cm}|p{13cm}|}
\hline
  \textbf{Question} && \textbf{Answer} \\
\hline
7a  & (1pt) &  \Paste{exsevena} \\ \hline
7b  & (1pt) &  \Paste{exsevenb} \\ \hline
8 & (1pt) & \Paste{exeight} \\ \hline
9a & (1pt) &  \Paste{exninea} \\ \hline 
9b & (1pt) &  \Paste{exnineb} \\ \hline
10a& (1pt) &  \Paste{extena} \\ \hline
10b& (1pt) &   \Paste{extenb} \\ \hline
11a& (1pt) &  \Paste{exelevena} \\ \hline
11b& (1pt) &   \Paste{exelevenb} \\ \hline
12& (1pt) &  \Paste{exetwelve} \\ \hline
\end{tabular}

\end{document}

