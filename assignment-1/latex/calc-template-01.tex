%
% Although we try to provide a template that completely
% matches the corresponding assignment, we do expect you
% to check that you have indeed answered all questions.
%

% ALSO VERY IMPORTANT:
% This is just a template to help you with the LaTeX part of the assignment.
% So you may change it completely according to your own wishes!
%

\documentclass[a4paper]{article}
% Typically the 'article' class is appropriate for assignments.
% And we print it on a4, so we include that as well.

\usepackage{a4wide}
% To decrease the margins and allow more text on a page.

\usepackage{graphicx}
% To deal with including pictures.

\usepackage{enumerate}
% To provide a little bit more functionality than with LaTeX's default
% enumerate environment.

\usepackage{array} 
% To provide a little bit more functionality than with LaTeX's default
% array environment.

\usepackage[american]{babel}
% Use this if you want to write the document in US English. It takes care of
% (usually) proper hyphenation.
% If you want to write your answers in Dutch, please replace 'american'
% by 'dutch'.
% Note that after a change it may be that the first compilation of LaTeX
% fails. That is normal and caused by the fact that in auxiliary files
% from previous runs, there may still be a \selectlanguage{american}
% around, which is invalid if 'american' is not incorporated with babel.

\usepackage{amssymb}
% This package loads mathematical things like the fonts for the blackboard
% bold for the set of natural numbers.

\usepackage{amsmath}
\usepackage{clipboard}

\newcommand{\exercise}[2]{\subsection*{Exercise #1}{#2}}
\newcommand{\exerciseenum}[2]{\subsection*{Exercise #1}{\begin{enumerate}[a)]#2\end{enumerate}}}
% We defined our own commands to make it easy to present all the
% exercises in the same style. The first one does not automatically
% start an 'enumerate' list, the second one does.
% The [2] means that our command needs two arguments.
% The #1 and the #2 indicate where we use these arguments in the
% command.
% There are several ways to have automatic numbering for the exercises,
% but here we have chosen to use a subsection for this and use manual
% numbering. This is because maybe not everyone will be able to do hand in
% all exercises.
% Note that we add the '*' to make sure that the subsection is not numbered.
% (Since we don't have a \section, the numbers for a subsection would be
% ugly like 0.1, 0.2 et cetera.
% The environment 'enumerate' automatically numbers the items in this list.
% The optional [a)] makes sure that the list will be like a), b), c) et cetera.


\newcommand{\set}[1]{\ensuremath{\left\{{#1}\right\}}}
% This command puts curly braces around its argument, so it becomes
% a set. The \left and \right make sure that the braces grow in size
% if the contents of the set are large symbols.

\newcommand{\setbuild}[2]{\ensuremath{\set{{#1}\mid{#2}}}}
% We also introduce a shortcut for using the set builder notation.
% Do you understand what it does?

% And the next series of commands gives you some of the default sets
% that were in the slides.
\newcommand{\TT}{\ensuremath{\mathbb{T}}}
\newcommand{\FF}{\ensuremath{\mathbb{F}}}
\newcommand{\NN}{\ensuremath{\mathbb{N}}}
\newcommand{\NNp}{\ensuremath{\mathbb{N}^{+}}}
\newcommand{\ZZ}{\ensuremath{\mathbb{Z}}}
\newcommand{\ZZp}{\ensuremath{\mathbb{Z}^{+}}}
\newcommand{\QQ}{\ensuremath{\mathbb{Q}}}
\newcommand{\QQp}{\ensuremath{\mathbb{Q}^{+}}}
\newcommand{\RR}{\ensuremath{\mathbb{R}}}
\newcommand{\RRp}{\ensuremath{\mathbb{R}^{+}}}
\newcommand{\CC}{\ensuremath{\mathbb{C}}}

% An the next command gives a shorthand for the power set of a given set.
\newcommand{\power}[1]{\ensuremath{{\cal P}\left({#1}\right)}}

\newcommand{\sol}[1]{\underline{\underline{#1}}}


\title{Calculus and Probability\\Assignment 1}

% Replace the placeholders by your real name, student number and 
% group (for the exercise hours)
\author{
\global\Copy{name}{
 Christoph Schmidl  % Fill in your name
} \\ 
\global\Copy{snumber}{
 s4226887  % Fill in your student number
} \\
\global\Copy{group}{
 Master Computing Science\\
 Group: Tutorial 5  % Fill in your assigned group
}}

% In LaTeX everything before \begin{document} is called pre-amble.
% This is where you put all important settings. The real document
% starts after \begin{document}.
\begin{document}
\maketitle 
% \maketitle makes sure that the title is shown on the first page of
% the document.


% Now we use the command we defined earlier and give it the proper two
% parameters.
% Because the second parameter is long, we put a % directly after the
% opening curly brace {. This is not needed but makes the source file
% look a bit better.

%\newpage

% #####################################################################################
%
% Exercises are below. They consist of a calculation part and a solution part. The solution
% part is automatically copied to the last page for correcting by the student asisstents.
% This is somewhat experimental.... If it doesn't work, copy-paste by hand. 
%
% #####################################################################################


\exerciseenum{6}{% 
\item%a
\begin{align*}
x - x^3 &= 0 \\
&= x(1 - x^2) = 0 \\
&\rightarrow x_1 = -1, x_2 = 0, x_3 = 1
\end{align*}
\global\Copy{exzesa}{
Values x for which $f(x) = 0 \rightarrow \{-1, 0, 1 \}$  % Put here the final answer, which is then also copied to the form on the last page
}

\item%b
\begin{align*}
x - x^3 > 0 \\
&= x(1 - x^2) > 0 \\
&\rightarrow (0, 1),(-\infty, -1)
\end{align*}
\global\Copy{exzesb}{
Values x for which $f(x) > 0 \rightarrow (0, 1),(-\infty, -1)$  % Put here the final answer
}

}

\exercise{7}{%

There are three cases in total.\\

\begin{enumerate}
	\item When $a = b$, then $y = \{ a \}$
	\item When $b > a$, then $(b - a) \geq 0$. $x \in (0,1) \rightarrow y = a + (b - a)x > a$ and $y = a + (b - a)x < a + (b - a) = b$. Determine if $y$ takes every value in the interval $(a,b)$: $c \in (a,b)$. $ c = a + (b - a)x_c \Leftrightarrow x_c = \frac{c - a}{b - a}$. Therefore $y$ runs through all values in $(a,b)$ because for every element $c$ in $(a,b)$ there exists an element $x_c \in (0,1)$ such that $c = a + (b - a)x_c$. Y runs through all values in $(a,b)$. 
	\item When $b < a$, then $(b - a) < 0$. Using $x \in (0,1)$, there is $y = a + (b - a)x < a$ and $y = a + (b - a)x > a + (b - a) = b \rightarrow $ y takes values in $(b,a)$. Using the same procedure as above, $y$ runs through all values in $(b,a)$.
\end{enumerate}


\global\Copy{exzeven}{
$y$ runs through all values in $(a,b)$, except when $a = b$, then $y = \{ a \}$
}
}

\exerciseenum{8}{%
\item%a
$f(-x) =  3(-x) - (-x)^3 = -3x + x^3 = -(3x - x^3) = -f(x)$. The function is odd.
\global\Copy{exachta}{
The function is odd.
}
\item%b
$f(-x) = \sqrt[3]{( 1 - (-x))^2} + \sqrt[3]{(1 + (-x))^2} = \sqrt[3]{(1 + x)^2} + \sqrt[3]{(1 - x)^2} = f(x)$. The function is even.
\global\Copy{exachtb}{
The function is even.
}
}

\exerciseenum{9}{%
\item%a

$f(x) = \sqrt{7 - x^2} + 1$\\
Assume that has to be $7 - x^2 \geq 0$. The only possible values lie in $- \sqrt{7} \leq x \leq \sqrt{7}$. So $D(f) = [- \sqrt{7}$\\

As $0 \leq 7 - x^2 \leq 7$, we have $0 \leq \sqrt{7 - x^2} \leq \sqrt{7}$, so $1 \leq f(x) \leq \sqrt{7} + 1$. So $R(f) = [1, \sqrt{7} + 1]$

\global\Copy{exnegena}{
$D(f) = [- \sqrt{7}, \sqrt{7}]$, $R(f) = [1, \sqrt{7} + 1]$
}
\item%b
$f(x) = \frac{1}{|x|}$\\
The domain only excludes $x = 0$. So $D(f) = \mathbb{R} \setminus \{ 0 \}$. The range is $R(f) = (0, \infty)$
\global\Copy{exnegenb}{
$D(f) = \mathbb{R} \setminus \{ 0 \}$, $R(f) = (0, \infty)$
}
}

\exerciseenum{10}{%
\item%a

$y(cx + d) = ax + b \rightarrow x(cy - a) = -dy + b$ so $x = \frac{-dy + b}{cy - a}$. So the inverse of y is $\frac{-dx + b}{cx - a}$

\global\Copy{extiena}{
The inverse of y is $\frac{-dx + b}{cx - a}$
}
\item%b

\global\Copy{extienb}{
It is equal to $y$ when $d = -a$
}
}

\exerciseenum{11}{%
\item%a
\begin{align*}
\lim_{x \to 2} \frac{x - 2}{ x^2 + x - 6} &= \lim_{x \to 2} \frac{x - 2}{ (x-2)(x+3)} \\
&= \lim_{x \to 2} \frac{1}{ (x+3)} \\
&= \frac{1}{5}
\end{align*}
 % Put here your calculation
\global\Copy{exelfa}{
$\lim_{x \to 2} \frac{x - 2}{ x^2 + x - 6} = \frac{1}{5}$ % Put here the final answer
}
\item%b
\begin{align*}
\lim_{x \to 1} \frac{x^2 - 4x + 3}{x^2 + x -2} &= \lim_{x \to 1} \frac{(x-1)(x-3)}{(x+2)(x-1)} \\
&= \lim_{x \to 1} \frac{(x-3)}{(x+2)} \\
&= - \frac{2}{3}
\end{align*}
\global\Copy{exelfb}{
$\lim_{x \to 1} \frac{x^2 - 4x + 3}{x^2 + x -2} = -\frac{2}{3}$
}
}


% If everything works correctly the page below is filled in automatically. If not, do some 
% copy-paste by hand. 

\newpage
\section*{Answer Form Assignment 1}

\noindent
\large
\begin{tabular}{|p{4cm}|p{11.5cm}|}
\hline
\textbf{Name} & \Paste{name} \\ \hline
\textbf{Student Number} & \Paste{snumber} \\ \hline
\end{tabular}

\vspace{0.5cm}
\noindent
\large
%\begin{tabular}{|m{2.5cm}|m{13cm}|}
\begin{tabular}{|m{1cm}@{}m{1.5cm}|p{13cm}|}
\hline
  \textbf{Question} && \textbf{Answer} \\
\hline
6a & (1pt) & \Paste{exzesa} \\  \hline
6b & (1pt) & \Paste{exzesb} \\  \hline
7  & (1pt) &  \Paste{exzeven} \\ \hline
8a & (0.5pt) & \Paste{exachta} \\ \hline
8b & (0.5pt) & \Paste{exachtb} \\ \hline
9a & (1pt) &  \Paste{exnegena} \\ \hline 
9b & (1pt) &  \Paste{exnegenb} \\ \hline
10a& (1pt) &  \Paste{extiena} \\ \hline
10b& (1pt) &   \Paste{extienb} \\ \hline
11a& (1pt) &   \Paste{exelfa} \\ \hline
11b& (1pt) &   \Paste{exelfb} \\ \hline
\end{tabular}



\end{document}

