%
% Although we try to provide a template that completely
% matches the corresponding assignment, we do expect you
% to check that you have indeed answered all questions.
%

% ALSO VERY IMPORTANT:
% This is just a template to help you with the LaTeX part of the assignment.
% So you may change it completely according to your own wishes!
%

\documentclass[a4paper]{article}
% Typically the 'article' class is appropriate for assignments.
% And we print it on a4, so we include that as well.

\usepackage{a4wide}
% To decrease the margins and allow more text on a page.

\usepackage{graphicx}
% To deal with including pictures.

\usepackage{enumerate}
% To provide a little bit more functionality than with LaTeX's default
% enumerate environment.

\usepackage{array} 
% To provide a little bit more functionality than with LaTeX's default
% array environment.

\usepackage[american]{babel}
% Use this if you want to write the document in US English. It takes care of
% (usually) proper hyphenation.
% If you want to write your answers in Dutch, please replace 'american'
% by 'dutch'.
% Note that after a change it may be that the first compilation of LaTeX
% fails. That is normal and caused by the fact that in auxiliary files
% from previous runs, there may still be a \selectlanguage{american}
% around, which is invalid if 'american' is not incorporated with babel.

\usepackage{amssymb}
% This package loads mathematical things like the fonts for the blackboard
% bold for the set of natural numbers.

\usepackage{amsmath}
\usepackage{clipboard}

\newcommand{\exercise}[2]{\subsection*{Exercise #1}{#2}}
\newcommand{\exerciseenum}[2]{\subsection*{Exercise #1}{\begin{enumerate}[a)]#2\end{enumerate}}}
% We defined our own commands to make it easy to present all the
% exercises in the same style. The first one does not automatically
% start an 'enumerate' list, the second one does.
% The [2] means that our command needs two arguments.
% The #1 and the #2 indicate where we use these arguments in the
% command.
% There are several ways to have automatic numbering for the exercises,
% but here we have chosen to use a subsection for this and use manual
% numbering. This is because maybe not everyone will be able to do hand in
% all exercises.
% Note that we add the '*' to make sure that the subsection is not numbered.
% (Since we don't have a \section, the numbers for a subsection would be
% ugly like 0.1, 0.2 et cetera.
% The environment 'enumerate' automatically numbers the items in this list.
% The optional [a)] makes sure that the list will be like a), b), c) et cetera.


\newcommand{\set}[1]{\ensuremath{\left\{{#1}\right\}}}
% This command puts curly braces around its argument, so it becomes
% a set. The \left and \right make sure that the braces grow in size
% if the contents of the set are large symbols.

\newcommand{\setbuild}[2]{\ensuremath{\set{{#1}\mid{#2}}}}
% We also introduce a shortcut for using the set builder notation.
% Do you understand what it does?

% And the next series of commands gives you some of the default sets
% that were in the slides.
\newcommand{\TT}{\ensuremath{\mathbb{T}}}
\newcommand{\FF}{\ensuremath{\mathbb{F}}}
\newcommand{\NN}{\ensuremath{\mathbb{N}}}
\newcommand{\NNp}{\ensuremath{\mathbb{N}^{+}}}
\newcommand{\ZZ}{\ensuremath{\mathbb{Z}}}
\newcommand{\ZZp}{\ensuremath{\mathbb{Z}^{+}}}
\newcommand{\QQ}{\ensuremath{\mathbb{Q}}}
\newcommand{\QQp}{\ensuremath{\mathbb{Q}^{+}}}
\newcommand{\RR}{\ensuremath{\mathbb{R}}}
\newcommand{\RRp}{\ensuremath{\mathbb{R}^{+}}}
\newcommand{\CC}{\ensuremath{\mathbb{C}}}

% An the next command gives a shorthand for the power set of a given set.
\newcommand{\power}[1]{\ensuremath{{\cal P}\left({#1}\right)}}

\newcommand{\sol}[1]{\underline{\underline{#1}}}


\title{Calculus and Probability\\Assignment 2}

% Replace the placeholders by your real name, student number and 
% group (for the exercise hours)
\author{
\global\Copy{name}{
 Christoph Schmidl  % Fill in your name
} \\ 
\global\Copy{snumber}{
 s4226887  % Fill in your student number
} \\
\global\Copy{group}{
 Master Computing Science\\Group: Tutorial 5  % Fill in your assigned group
}}

% In LaTeX everything before \begin{document} is called pre-amble.
% This is where you put all important settings. The real document
% starts after \begin{document}.
\begin{document}
\maketitle 
% \maketitle makes sure that the title is shown on the first page of
% the document.


% Now we use the command we defined earlier and give it the proper two
% parameters.
% Because the second parameter is long, we put a % directly after the
% opening curly brace {. This is not needed but makes the source file
% look a bit better.

%\newpage

% #####################################################################################
%
% Exercises are below. They consist of a calculation part and a solution part. The solution
% part is automatically copied to the last page for correcting by the student asisstents.
% This is somewhat experimental.... If it doesn't work, copy-paste by hand. 
%
% #####################################################################################

\exerciseenum{6}{% 
\item%a

By dividing the numerator and denominator by $x^3$ we get $\frac{x^3 + 2x^2 + 2}{3x^3 + x + 4} = \frac{1 + \frac{2}{x} + \frac{2}{x^3}}{3 + \frac{1}{x^2} + \frac{4}{x^3}}$. By ignoring the last terms of both the numerator and the denominator because they do not contribute that much in the very end, we get 

\global\Copy{exsixa}{
\begin{align*}
\lim_{x \to - \infty} \frac{x^3 + 2x^2 + 2}{3x^3 + x + 4} = \frac{1}{3}
\end{align*}
}

\item%b

By dividing the numerator and denominator by $x^2$ we get $\frac{2x+1}{x^2+x} = \frac{\frac{2}{x}+\frac{1}{x^2}}{1 + \frac{1}{x}}$. The numerator converges towards 0 and the denominator towards 1. Therefore we get

\global\Copy{exsixb}{
  \begin{align*}
  \lim_{x \to \infty} \frac{2x+1}{x^2+x} = 0
  \end{align*}
}

}

\exerciseenum{7}{% 
\item%a

\begin{align*}
f(a + h) = 2(a + h) + 3 = 2a + 2h + 3
\end{align*}

\begin{align*}
f(a) = 2a + 3
\end{align*}

\begin{align*}
f(a + h) - f(a) = 2a + 2h + 3 - 2a - 3 = 2h
\end{align*}


\global\Copy{exsevena}{

\begin{align*}
f'(a) = \lim_{h \to 0} \frac{f(a+h) - f(a)}{h} = \lim_{h \to 0} \frac{2h}{2} = 2
\end{align*}

}

\item%b

\begin{align*}
f(a+h) = \frac{5(a+h)-7}{f(a+h)+3} = \frac{5a-7+5h}{4a+3+4h}
\end{align*}

\begin{align*}
f(a+h) - f(a) &= \frac{5a - 7 + 5h}{4a + 3 + 4h} - \frac{5a - 7}{4a + 3}\\
&= \frac{(5a-7+5h)(4a+3)-(5a-7)(4a+3+4h)}{(4a + 3 + 4h)(4a + 3)}\\
&= \frac{43h}{(4a+3+4h)(4a+3)}
\end{align*}

\begin{align*}
\lim_{h \to 0} \frac{f(a+h)-f(a)}{h} &= \lim_{h \to 0} \frac{43h}{(4a+3+4h)(4a+3)} \frac{1}{h}\\
&= \lim_{h \to 0} \frac{43}{(4a+3+4h)(4a+3)}\\
&= \frac{43}{(4a+3)^2}
\end{align*}


\global\Copy{exsevenb}{
$f'(a) = \frac{43}{(4a+3)^2}$ for every $a$ in $\mathbb{R} \setminus \{ - \frac{3}{4}\}$
}

}

\exerciseenum{8}{% 
\item%a

Getting the slope of the tangent line at $x = 2$:

\begin{align*}
	f'(x) = \frac{1}{(x+1)^2}\\
	f'(2) = \frac{1}{9}
\end{align*}

Getting a point on the tangent line to be able to formulate the equation:

\begin{align*}
f(x) = \frac{1}{1 + \frac{1}{2}} \rightarrow  y = \frac{2}{3}
\end{align*}

Therefore, we now have found the coordinate $(2, \frac{2}{3})$ for the point shared by $f(x)$ and the line to $f(x) = 2$. The only step left is to use the point $(2, \frac{2}{3})$ and slope $\frac{1}{9}$ in the point-slope formula for a line:

\begin{align*}
y - y_1 &= (m(x - x_1))\\
y &= \frac{x}{9} + \frac{4}{9}
\end{align*}


\global\Copy{exeighta}{

\begin{align*}
y = \frac{x}{9} + \frac{4}{9}
\end{align*}

}

\item%b
\dots  % Put here your calculation
\global\Copy{exeightb}{
  Answer 8b% Put here the final answer, which is then also copied to the form on the last page
}

}

\exerciseenum{9}{% 
\item%a

$(e^x)' = e^x$, $(tan(x))' = \frac{1}{cos^2 x}$ Using the chain rule we get

\global\Copy{exninea}{

\begin{align*}
f'(x) = exp(tan(x))\frac{1}{cos^2x} = \frac{exp(tan(x))}{cos^2x}
\end{align*}

}

\item%b

$(\ln x)' = \frac{1}{x}$, $(\cos x))' = - \sin x$ Using the chain rule we get

\global\Copy{exnineb}{
\begin{align*}
f'(x) = - \frac{1}{\cos x}(- \sin x) = \frac{\sin x}{\cos x} = tan(x)
\end{align*}

}

}

\exerciseenum{10}{% 
\item%a

1. Using the chain rule given that $g(x) = x^x$ and $h(x) = exp \; x \rightarrow f(x) = (g \circ h)(x)$. $g'(x) = x^x(log \; x + 1)$, $h'(x) = exp \; x$. Therefore:

\begin{align*}
f'(x) = g'(h(x))h'(x) = (exp \; x)^{exp \; x}(x + 1)exp \; x = (exp \; x)^{exp \; x+1}(x+1) = e^{x(e^x + 1)}(x+1)
\end{align*}

2. Using logarithmic differentiation: $\ln((exp \; x)^{exp \; x}) = x \; exp \; x \rightarrow \frac{f'(x)}{f(x)} = exp \; x(x+1)$.\\

\begin{align*}
f'(x) = (exp \; x)^{exp \; x+1}(x+1) = e^{x(e^x+1)}(x+1)
\end{align*}

\global\Copy{extena}{
\begin{align*}
f'(x) = (exp \; x)^{exp \; x+1}(x+1) = e^{x(e^x+1)}(x+1)
\end{align*}
}

\item%b
1. Using the inverse: $f^{-1}(x) = x^2 + 2 \rightarrow (f^{-1})'(x) = 2x$\\
2. $f^{-1}(x) = x^2 + 2$, $f'(x) = \frac{1}{2 \sqrt{x-2}} \rightarrow (f^{-1})'(x) = \frac{1}{f'(f^{-1}(x))} ? \frac{1}{\frac{1}{2 \sqrt{(x^2+2)-2}}} = \frac{1}{\frac{1}{2x}} = 2x$
\global\Copy{extenb}{
\begin{align*}
(f^{-1})'(x) = 2x
\end{align*}
}

}


% If everything works correctly the page below is filled in automatically. If not, do some 
% copy-paste by hand. 

\newpage
\section*{Answer Form Assignment 2}

\noindent
\large
\begin{tabular}{|p{4cm}|p{11.5cm}|}
\hline
\textbf{Name} & \Paste{name} \\ \hline
\textbf{Student Number} & \Paste{snumber} \\ \hline
\end{tabular}

\vspace{0.5cm}
\noindent
\large
%\begin{tabular}{|m{2.5cm}|m{13cm}|}
\begin{tabular}{|m{1cm}@{}m{1.5cm}|p{13cm}|}
\hline
  \textbf{Question} && \textbf{Answer} \\
\hline
6a & (1pt) & \Paste{exsixa} \\  \hline
6b & (1pt) & \Paste{exsixb} \\  \hline
7a  & (1pt) &  \Paste{exsevena} \\ \hline
7b  & (1pt) &  \Paste{exsevenb} \\ \hline
8a & (1pt) & \Paste{exeighta} \\ \hline
8b & (1pt) & \Paste{exeightb} \\ \hline
9a & (1pt) &  \Paste{exninea} \\ \hline 
9b & (1pt) &  \Paste{exnineb} \\ \hline
10a& (1pt) &  \Paste{extena} \\ \hline
10b& (1pt) &   \Paste{extenb} \\ \hline
\end{tabular}

\end{document}

