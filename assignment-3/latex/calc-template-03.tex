%
% Although we try to provide a template that completely
% matches the corresponding assignment, we do expect you
% to check that you have indeed answered all questions.
%

% ALSO VERY IMPORTANT:
% This is just a template to help you with the LaTeX part of the assignment.
% So you may change it completely according to your own wishes!
%

\documentclass[a4paper]{article}
% Typically the 'article' class is appropriate for assignments.
% And we print it on a4, so we include that as well.

\usepackage{a4wide}
% To decrease the margins and allow more text on a page.

\usepackage{graphicx}
% To deal with including pictures.

\usepackage{enumerate}
% To provide a little bit more functionality than with LaTeX's default
% enumerate environment.

\usepackage{array} 
% To provide a little bit more functionality than with LaTeX's default
% array environment.

\usepackage[american]{babel}
% Use this if you want to write the document in US English. It takes care of
% (usually) proper hyphenation.
% If you want to write your answers in Dutch, please replace 'american'
% by 'dutch'.
% Note that after a change it may be that the first compilation of LaTeX
% fails. That is normal and caused by the fact that in auxiliary files
% from previous runs, there may still be a \selectlanguage{american}
% around, which is invalid if 'american' is not incorporated with babel.

\usepackage{amssymb}
% This package loads mathematical things like the fonts for the blackboard
% bold for the set of natural numbers.

\usepackage{amsmath}
\usepackage{clipboard}

\newcommand{\exercise}[2]{\subsection*{Exercise #1}{#2}}
\newcommand{\exerciseenum}[2]{\subsection*{Exercise #1}{\begin{enumerate}[a)]#2\end{enumerate}}}
% We defined our own commands to make it easy to present all the
% exercises in the same style. The first one does not automatically
% start an 'enumerate' list, the second one does.
% The [2] means that our command needs two arguments.
% The #1 and the #2 indicate where we use these arguments in the
% command.
% There are several ways to have automatic numbering for the exercises,
% but here we have chosen to use a subsection for this and use manual
% numbering. This is because maybe not everyone will be able to do hand in
% all exercises.
% Note that we add the '*' to make sure that the subsection is not numbered.
% (Since we don't have a \section, the numbers for a subsection would be
% ugly like 0.1, 0.2 et cetera.
% The environment 'enumerate' automatically numbers the items in this list.
% The optional [a)] makes sure that the list will be like a), b), c) et cetera.


\newcommand{\set}[1]{\ensuremath{\left\{{#1}\right\}}}
% This command puts curly braces around its argument, so it becomes
% a set. The \left and \right make sure that the braces grow in size
% if the contents of the set are large symbols.

\newcommand{\setbuild}[2]{\ensuremath{\set{{#1}\mid{#2}}}}
% We also introduce a shortcut for using the set builder notation.
% Do you understand what it does?

% And the next series of commands gives you some of the default sets
% that were in the slides.
\newcommand{\TT}{\ensuremath{\mathbb{T}}}
\newcommand{\FF}{\ensuremath{\mathbb{F}}}
\newcommand{\NN}{\ensuremath{\mathbb{N}}}
\newcommand{\NNp}{\ensuremath{\mathbb{N}^{+}}}
\newcommand{\ZZ}{\ensuremath{\mathbb{Z}}}
\newcommand{\ZZp}{\ensuremath{\mathbb{Z}^{+}}}
\newcommand{\QQ}{\ensuremath{\mathbb{Q}}}
\newcommand{\QQp}{\ensuremath{\mathbb{Q}^{+}}}
\newcommand{\RR}{\ensuremath{\mathbb{R}}}
\newcommand{\RRp}{\ensuremath{\mathbb{R}^{+}}}
\newcommand{\CC}{\ensuremath{\mathbb{C}}}

% An the next command gives a shorthand for the power set of a given set.
\newcommand{\power}[1]{\ensuremath{{\cal P}\left({#1}\right)}}

\newcommand{\sol}[1]{\underline{\underline{#1}}}


\title{Calculus and Probability\\Assignment 3}

% Replace the placeholders by your real name, student number and 
% group (for the exercise hours)
\author{
\global\Copy{name}{
 Christoph Schmidl  % Fill in your name
} \\ 
\global\Copy{snumber}{
 s4226887  % Fill in your student number
} \\
\global\Copy{group}{
 Master Computing Science\\Group: Tutorial 5  % Fill in your assigned group
}}

% In LaTeX everything before \begin{document} is called pre-amble.
% This is where you put all important settings. The real document
% starts after \begin{document}.
\begin{document}
\maketitle 
% \maketitle makes sure that the title is shown on the first page of
% the document.


% Now we use the command we defined earlier and give it the proper two
% parameters.
% Because the second parameter is long, we put a % directly after the
% opening curly brace {. This is not needed but makes the source file
% look a bit better.

%\newpage

% #####################################################################################
%
% Exercises are below. They consist of a calculation part and a solution part. The solution
% part is automatically copied to the last page for correcting by the student asisstents.
% This is somewhat experimental.... If it doesn't work, copy-paste by hand. 
%
% #####################################################################################

\exerciseenum{6}{% 
\item%a
\begin{align*}
f(x) &= \arccos(\cos(x^2)) \\
f'(x) &= \frac{2x \sin(x^2)}{\sqrt{1 - \cos^2(x^2)}}
\end{align*}
\global\Copy{exsix}{
$f'(x) = \frac{2x \sin(x^2)}{\sqrt{1 - \cos^2(x^2)}}$
}

}

\exerciseenum{7}{% 
\item%a
The numerator outweighs the denominator if we look to the limit towards infinity. Therefore:
\global\Copy{exsevena}{
\begin{align*}
\lim_{x \to \infty} \frac{x^2}{1 + e^{-x}} = \infty
\end{align*}
}

\item%b

We get $\frac{0}{0}$ and therefore can apply L'Hopital which gives us $\lim_{x \to 0} \frac{\cos(x) + A + 3Bx^2}{5x^4} = \frac{1+A}{0} \rightarrow \pm \infty$ unless $A = -1$

\global\Copy{exsevenb}{
After applying L'Hopital repeatedly, we get $A \cdot B \cdot C = -1 \cdot \frac{1}{6} \cdot 120 = -20$
}

}

\exerciseenum{8}{% 
\item%a

\begin{enumerate}
	\item $g(x) = \cos(3x)$
	\item $g^{(1)}(x) = - 3 \sin(3x)$
	\item $g^{(2)}(x) = -3^2 \cos(3x)$
	\item $g^{(3)}(x) = 3^3 \sin(3x)$
	\item $g^{(4)}(x) = 3^4 \cos(3x)$
\end{enumerate}

\vspace{1em}

Because the remainder of 2015 divided by 4 is 3, we get:



\global\Copy{exeight}{
\begin{align*}
g^{2015}(x) = 3^{2015} \sin(3x)
\end{align*}
}


}

\exerciseenum{9}{% 
\item%a

There are two roots which you can see right from the term without further calculation, namely:

\begin{align*}
x_0 = -1 \quad x_1 = 3
\end{align*}

We get the y-intercept by putting a zero into the function, therefore:

\begin{align*}
f(0) = -3 
\end{align*}

\global\Copy{exninea}{
Two roots: $x_0 = -1$ and $x_1 = 3$. y-intercept: $f(0) = -3$
}

\item%b

$(x+1)^2$ goes into the positive direction, therefore the whole term depends on $(x - 3)$. Therefore we get:

\global\Copy{exnineb}{
\begin{align*}
\lim_{x \ to - \infty} = - \infty \quad \text{and} \quad \lim_{x \ to + \infty} = + \infty
\end{align*}
}

\item%c

In order to find the local minima and maxima we need to calculate the first and the second derivative of the given function. After that we have to find the zeros of the first and second derivative in order to get the critical points. The critical points can then be tested for the requirements of being local maxima or minima.

\begin{align*}
f(x) &= (x+1)^2(x-3) \\
f'(x) &= ((x^2 + 2x + 1)(x-3)) = (2x+2)(x-3) + x^2 + 2x + 1 = 3x^2 - 2x - 5\\
f''(x) &= 6x - 2
\end{align*}

The product rule has been applied to get the first derivative.\\

Zeros of of $f'(x) = 3x^2 - 2x - 5 = (x+1)(3x-5)$ are $x_0 = -1$ and $x_1 = \frac{5}{3}$\\
Zeros of $f''(x) = 6x - 2$ are $x = \frac{1}{3}$\\

We already got the x-coordinates of the critical points and only have to plug them in into the original function in order to get the y-coordinates:\\

\begin{align*}
f(-1) = 0\\
f(\frac{5}{3}) \approx - 9.481
\end{align*}

Therefore, we get the critical points $(-1,0)$ and $(\frac{5}{3}, - 9.481)$\\

 
\global\Copy{exninec}{
$f''(-1) = -14 < 0$, therefore it is a maximum and $f''(\frac{5}{3}) = 13 > 0$, therefore is is a minimum.
}

\item%d

\global\Copy{exnined}{
The function is concave $\Leftrightarrow f''(x) < 0 \Leftrightarrow 6x - 2 < 0 \Leftrightarrow x < \frac{1}{3}$. The function is convex $\Leftrightarrow f''(x) > 0 \Leftrightarrow 6x - 2 > 0 \Leftrightarrow x > \frac{1}{3}$. The point of inflection is at $x = \frac{1}{3}$
}

}

\exerciseenum{10}{% 
\item%a
\global\Copy{extena}{
$\frac{1}{6} \sin(x)(\cos(2x)+5) + C$
}

\item%b

\global\Copy{extenb}{
$f_i(x) = \frac{1}{2} \sin^2(x) + C_i$ and pick three distinct value for $C_i$
}

}


% If everything works correctly the page below is filled in automatically. If not, do some 
% copy-paste by hand. 

\newpage
\section*{Answer Form Assignment 3}

\noindent
\large
\begin{tabular}{|p{4cm}|p{11.5cm}|}
\hline
\textbf{Name} & \Paste{name} \\ \hline
\textbf{Student Number} & \Paste{snumber} \\ \hline
\end{tabular}

\vspace{0.5cm}
\noindent
\large
%\begin{tabular}{|m{2.5cm}|m{13cm}|}
\begin{tabular}{|m{1cm}@{}m{1.5cm}|p{13cm}|}
\hline
  \textbf{Question} && \textbf{Answer} \\
\hline
6 & (1pt) & \Paste{exsix} \\  \hline
7a  & (1pt) &  \Paste{exsevena} \\ \hline
7b  & (1pt) &  \Paste{exsevenb} \\ \hline
8a & (1pt) & \Paste{exeight} \\ \hline
9a & (1pt) &  \Paste{exninea} \\ \hline 
9b & (1pt) &  \Paste{exnineb} \\ \hline
9c & (1pt) &  \Paste{exninec} \\ \hline 
9d & (1pt) &  \Paste{exnined} \\ \hline
10a& (1pt) &  \Paste{extena} \\ \hline
10b& (1pt) &   \Paste{extenb} \\ \hline
\end{tabular}

\end{document}

