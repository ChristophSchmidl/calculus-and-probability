%
% Although we try to provide a template that completely
% matches the corresponding assignment, we do expect you
% to check that you have indeed answered all questions.
%

% ALSO VERY IMPORTANT:
% This is just a template to help you with the LaTeX part of the assignment.
% So you may change it completely according to your own wishes!
%

\documentclass[a4paper]{article}
% Typically the 'article' class is appropriate for assignments.
% And we print it on a4, so we include that as well.

\usepackage{a4wide}
% To decrease the margins and allow more text on a page.

\usepackage{graphicx}
% To deal with including pictures.

\usepackage{enumerate}
% To provide a little bit more functionality than with LaTeX's default
% enumerate environment.

\usepackage{array} 
% To provide a little bit more functionality than with LaTeX's default
% array environment.

\usepackage[american]{babel}
% Use this if you want to write the document in US English. It takes care of
% (usually) proper hyphenation.
% If you want to write your answers in Dutch, please replace 'american'
% by 'dutch'.
% Note that after a change it may be that the first compilation of LaTeX
% fails. That is normal and caused by the fact that in auxiliary files
% from previous runs, there may still be a \selectlanguage{american}
% around, which is invalid if 'american' is not incorporated with babel.

\usepackage{amssymb}
% This package loads mathematical things like the fonts for the blackboard
% bold for the set of natural numbers.

\usepackage{amsmath}
\usepackage{clipboard}

\newcommand{\exercise}[2]{\subsection*{Exercise #1}{#2}}
\newcommand{\exerciseenum}[2]{\subsection*{Exercise #1}{\begin{enumerate}[a)]#2\end{enumerate}}}
% We defined our own commands to make it easy to present all the
% exercises in the same style. The first one does not automatically
% start an 'enumerate' list, the second one does.
% The [2] means that our command needs two arguments.
% The #1 and the #2 indicate where we use these arguments in the
% command.
% There are several ways to have automatic numbering for the exercises,
% but here we have chosen to use a subsection for this and use manual
% numbering. This is because maybe not everyone will be able to do hand in
% all exercises.
% Note that we add the '*' to make sure that the subsection is not numbered.
% (Since we don't have a \section, the numbers for a subsection would be
% ugly like 0.1, 0.2 et cetera.
% The environment 'enumerate' automatically numbers the items in this list.
% The optional [a)] makes sure that the list will be like a), b), c) et cetera.


\newcommand{\set}[1]{\ensuremath{\left\{{#1}\right\}}}
% This command puts curly braces around its argument, so it becomes
% a set. The \left and \right make sure that the braces grow in size
% if the contents of the set are large symbols.

\newcommand{\setbuild}[2]{\ensuremath{\set{{#1}\mid{#2}}}}
% We also introduce a shortcut for using the set builder notation.
% Do you understand what it does?

% And the next series of commands gives you some of the default sets
% that were in the slides.
\newcommand{\TT}{\ensuremath{\mathbb{T}}}
\newcommand{\FF}{\ensuremath{\mathbb{F}}}
\newcommand{\NN}{\ensuremath{\mathbb{N}}}
\newcommand{\NNp}{\ensuremath{\mathbb{N}^{+}}}
\newcommand{\ZZ}{\ensuremath{\mathbb{Z}}}
\newcommand{\ZZp}{\ensuremath{\mathbb{Z}^{+}}}
\newcommand{\QQ}{\ensuremath{\mathbb{Q}}}
\newcommand{\QQp}{\ensuremath{\mathbb{Q}^{+}}}
\newcommand{\RR}{\ensuremath{\mathbb{R}}}
\newcommand{\RRp}{\ensuremath{\mathbb{R}^{+}}}
\newcommand{\CC}{\ensuremath{\mathbb{C}}}

% An the next command gives a shorthand for the power set of a given set.
\newcommand{\power}[1]{\ensuremath{{\cal P}\left({#1}\right)}}

\newcommand{\sol}[1]{\underline{\underline{#1}}}


\title{Calculus and Probability\\Assignment 4}

% Replace the placeholders by your real name, student number and 
% group (for the exercise hours)
\author{
\global\Copy{name}{
 Christoph Schmidl  % Fill in your name
} \\ 
\global\Copy{snumber}{
 s4226887  % Fill in your student number
} \\
\global\Copy{group}{
 Master Computing Science\\Group: Tutorial 5  % Fill in your assigned group
}}

% In LaTeX everything before \begin{document} is called pre-amble.
% This is where you put all important settings. The real document
% starts after \begin{document}.
\begin{document}
\maketitle 
% \maketitle makes sure that the title is shown on the first page of
% the document.


% Now we use the command we defined earlier and give it the proper two
% parameters.
% Because the second parameter is long, we put a % directly after the
% opening curly brace {. This is not needed but makes the source file
% look a bit better.

%\newpage

% #####################################################################################
%
% Exercises are below. They consist of a calculation part and a solution part. The solution
% part is automatically copied to the last page for correcting by the student asisstents.
% This is somewhat experimental.... If it doesn't work, copy-paste by hand. 
%
% #####################################################################################

\exerciseenum{6}{% 
\item%a
\dots  % Put here your calculation
\global\Copy{exsixa}{
  Answer 6a% Put here the final answer, which is then also copied to the form on the last page
}

\item%b
\dots  % Put here your calculation
\global\Copy{exsixb}{
  Answer 6b% Put here the final answer, which is then also copied to the form on the last page
}

}

% ########################

\exerciseenum{7}{% 
\item%a


\global\Copy{exsevena}{
\begin{enumerate}
	\item[i] $\frac{\partial}{\partial x} f(x,y) = - \sin (4y - xy)(-y) = y \sin (4y - xy)$
	\item[ii] $\frac{\partial}{\partial y} f(x,y) = - \sin (4y - xy)(4 - x)$
\end{enumerate}
}

\item%b
\global\Copy{exsevenb}{
\begin{enumerate}
	\item[i] $\frac{\partial}{\partial x} f(x,y) = \frac{e^\frac{x}{y}}{y}$
	\item[ii] $\frac{\partial}{\partial y} f(x,y) = - \frac{x e^\frac{x}{y}}{y^2}$
\end{enumerate}
}

}

% ########################

\exerciseenum{8}{% 
\item%a

\begin{align*}
\int_1^3 (3 \sqrt{x} + \frac{3}{x^2}) dx &= 3 \Big( \int_1^2 \sqrt{x} dx + \int_1^2 \frac{1}{x^2} dx \Big)\\
&= 3 \Big( \frac{2}{3} (2 \sqrt{2} - 1) + \frac{1}{2}\Big)\\
&= 4 \sqrt{2} - \frac{1}{2}
\end{align*}

\global\Copy{exeighta}{
\begin{align*}
\int_1^3 (3 \sqrt{x} + \frac{3}{x^2}) dx = 4 \sqrt{2} - \frac{1}{2}
\end{align*}


}

\item%b

\begin{align*}
\int_{-1}^1 \frac{-5}{\sqrt{1 - x^2}} dx &= -5 \Big( \int_{-1}^1 \frac{1}{\sqrt{1 - x^2}} dx \Big)\\
&= -5 \Big[ \arcsin(x) \Big]_{-1}^1 \\
&= -5 \Big[ \arcsin(1) - \arcsin(-1) \Big]\\
&= -5 \Big[ \frac{\pi}{2} - (-\frac{\pi}{2})\Big]\\
&= -5 \frac{2 \pi}{2}\\
&= - 5\pi
\end{align*}

\global\Copy{exeightb}{
\begin{align*}
\int_{-1}^1 \frac{-5}{\sqrt{1 - x^2}} dx = - 5\pi
\end{align*}
}

}

% ########################

\exerciseenum{9}{% 
\item%a

\begin{align*}
\int_{-\infty}^\frac{-\pi}{2} \frac{x \cos(x) - \sin(x)}{x^2} dx &= \lim_{b \to \infty} \Big( \Big[ \frac{\sin((x)}{x}\Big]_{-b}^\frac{-\pi}{2}\Big)\\
&= \lim_{b \to \infty} \Big( \frac{-1}{\frac{-\pi}{2}} - \frac{\sin(b)}{b}\Big)\\
&= \frac{2}{\pi}
\end{align*}

\global\Copy{exninea}{
\begin{align*}
\int_{-\infty}^\frac{-\pi}{2} \frac{x \cos(x) - \sin(x)}{x^2} dx = \frac{2}{\pi}
\end{align*}
}

\item%b
\dots  % Put here your calculation
\global\Copy{exnineb}{
  Answer 9b% Put here the final answer, which is then also copied to the form on the last page
}

}

% ########################

\exerciseenum{10}{% 
\item%a
\dots  % Put here your calculation
\global\Copy{extena}{
  Answer 10a% Put here the final answer, which is then also copied to the form on the last page
}

\item%b
\dots  % Put here your calculation
\global\Copy{extenb}{
  Answer 10b% Put here the final answer, which is then also copied to the form on the last page
}

}
% If everything works correctly the page below is filled in automatically. If not, do some 
% copy-paste by hand. 

\newpage
\section*{Answer Form Assignment 4}

\noindent
\large
\begin{tabular}{|p{4cm}|p{11.5cm}|}
\hline
\textbf{Name} & \Paste{name} \\ \hline
\textbf{Student Number} & \Paste{snumber} \\ \hline
\textbf{Group} & \Paste{group} \\ \hline
\end{tabular}

\vspace{0.5cm}
\noindent
\large
%\begin{tabular}{|m{2.5cm}|m{13cm}|}
\begin{tabular}{|m{1cm}@{}m{1.5cm}|p{13cm}|}
\hline
  \textbf{Question} && \textbf{Answer} \\
\hline

6a  & (1pt) &  \Paste{exsixa} \\ \hline
6b  & (1pt) &  \Paste{exsixb} \\ \hline
7a  & (1pt) &  \Paste{exsevena} \\ \hline
7b  & (1pt) &  \Paste{exsevenb} \\ \hline
8a  & (1pt) &  \Paste{exeighta} \\ \hline
8b  & (1pt) &  \Paste{exeightb} \\ \hline
9a  & (1pt) &  \Paste{exninea} \\ \hline
9b  & (1pt) &  \Paste{exnineb} \\ \hline
10a  & (1pt) &  \Paste{extena} \\ \hline
10b  & (1pt) &  \Paste{extenb} \\ \hline

\end{tabular}

\end{document}

