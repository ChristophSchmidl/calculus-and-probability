\documentclass[a4paper]{article}

\usepackage[english]{babel}
\usepackage{amsmath}
\usepackage{amssymb}
\usepackage{dsfont}
\usepackage{tikz}
\usepackage{framed} 
\usetikzlibrary{arrows,automata}
\title{Calculus and Probability Theory\\ Assignment 1}
\author{Christoph Schmidl\\
s4226887\\
Informatica\\
c.schmidl@student.ru.nl\\}
\date{\today}


\begin{document}
\maketitle

\textbf{After completing these exercises successfully you should be confident with the following topics:}

\begin{itemize}
	\item The domain and range of a function
	\item The limit of a function
\end{itemize}
\vspace{1em}

\begin{enumerate}

\item (\textbf{10 points}) Let $f(x) = x - x^3$. Determine the values $x$ for which


\begin{enumerate}
	\item[1.] $f(x) = 0$\\
	\textbf{Solution:}\\
	
	$\{-1,0,1\}$
		
	\item[2.] $f(x) > 0$\\
	\textbf{Solution:}\\
	
	$(0,1)$ and $(-\infty, -1)$
\end{enumerate}


\item (\textbf{10 points}) Let's assume that $x$ runs through the interval $(0,1)$. What values does $y$ run through for $y = a + (b - a)x$, where $a,b \in \mathbb{R}$

\textbf{Solution:}\\

The most useful form of straight-line equations is the "slope-intercept" form

\begin{equation}
	y = mx + b \notag
\end{equation}

where $m$ is the slope and $b$ gives the y-intercept.

In the given equation 

\begin{equation}
	y = (b-a)x + a \notag
\end{equation}

$(b-a)$ is the slope and $a$ is the y-intercept. Because we are dealing with variables here, I can't give definitive values for which y would run through. Nevertheless, the y-intercept would be $a$ and the slope would be $(b-a)$.


\item (\textbf{10 points}) Are the following function even or odd? In your explanation use the definition.\\


\fbox{\parbox{\textwidth}{\textit{Definition: Parity of function}\\

A function $f: (-a,a) \rightarrow \mathbb{R}$ is \textbf{even} if $f(-x) = f(x)$, for all $x \in (-a,a)$, and \textbf{odd} if $f(-x) = -f(x)$, for all $x \in (-a,a)$.}}


\begin{enumerate}
	\item $f(x) = 3x - x^3$\\
	\textbf{Solution:}\\
	
	Let's check if the function is odd by checking if $f(-x) = -f(x)$\\
	\begin{align}
		f(x) &= 3x - x^3 \notag\\
		f(-x) &= 3(-x) - (-x)^3 \notag\\
		f(-x) &= -3x + x^3 \notag\\
		&= -(3x - x^3) \notag\\
		&= -f(x) \notag
	\end{align}
	
	Therefore, the function is odd.
	\item $f(x) = \sqrt[3]{(1-x)^2} + \sqrt[3]{(1+x)^2}$\\
	\textbf{Solution:}\\
	
	Let's check if the function is even by checking if $f(-x) = f(x)$\\
	\begin{align}
		f(x) &= \sqrt[3]{(1-x)^2} + \sqrt[3]{(1+x)^2} \notag\\
		f(-x) &= \sqrt[3]{(1 - (-x))^2} + \sqrt[3]{(1 + (-x))^2} \notag\\
		f(-x) &= \sqrt[3]{(1 + x)^2} + \sqrt[3]{(1 - x)^2} \notag\\
		f(-x) &= f(x) \notag
	\end{align}	
	
	
	Therefore, the function is even.
\end{enumerate}


\item (\textbf{10 points}) What is the inverse of 

\begin{equation}
	y = \frac{ax + b}{cx + d} \; \; (ad - bc \neq 0)? \notag
\end{equation}
\textbf{Solution:}\\

\begin{itemize}
	\item Switch the x's for y's
	\begin{align*}
	\frac{ay - b}{cy + d} = x \notag\\
	\end{align*}	
	\item Multiply both sides by $cy+d$
	\begin{align*}
	\frac{ay - b}{cy + d} \times cy + d &= x  \times cy + d\notag\\
	ay -b &= x(cy+d)\notag
	\end{align*}
	\item Expand the x through the paranthesis
	\begin{align*}
	ay-b = cxy + dx\notag
	\end{align*}
	\item Move y's to one side
	\begin{align*}
	ay-cxy = b + dx\notag
	\end{align*}
	\item Factor out the x
	\begin{align*}
	y(a - cx) = b + dx\notag
	\end{align*}
	\item Divide both sides by $a - cx$
	\begin{align*}
	y = \frac{b+dx}{a - cx}\notag
	\end{align*}
\end{itemize}

The inverse is

\begin{equation}
	y = \frac{b + dx}{a - cx}\notag
\end{equation}



\item (\textbf{30 points}) Determine the domains and ranges of the following functions.

\begin{enumerate}
	\item $f(x) = \sqrt{7-x^2} + 1$\\
	\textbf{Solution:}\\
	
\begin{itemize}
	\item $D(f) = \{x \in \mathbb{R}| - \sqrt{7} \leq x \leq \sqrt{7}\}$ or in Interval-Notation: $[-\sqrt{7}, \sqrt{7}]$\\ 
	\item $R(f) = \{ f \in \mathbb{R} | 1 \leq f \leq 1 + \sqrt{7}\}$ or in Interval-Notation: \\$[1, 1 + \sqrt{7}]$\\
\end{itemize}	
	
	
	\item $f(x) = \frac{x-5}{x^2-3x-10}$\\
	\textbf{Solution:}\\
	
Trying to simplify the term:

\begin{align*}
	f(x) &= \frac{x-5}{x^2-3x-10}\notag\\
	&= \frac{x-5}{(x-5)(x+2)}\notag\\
	&= \frac{1}{x+2}\notag
\end{align*}	
	
\begin{itemize}
	\item $D(f) = \{x \in \mathbb{R}| x \neq -2\}$ or in Interval-Notation: $(-\infty,-2) \cup (-2, \infty)$\\ 
	\item $R(f) = \{ f \in \mathbb{R} | f \neq 0\}$ or in Interval-Notation: \\$(-\infty,0) \cup (0,\infty)$\\
\end{itemize}	
	
	
	\item $f(x) = \frac{1}{|x|}$\\
	\textbf{Solution:}\\
	
\begin{itemize}
	\item $D(f) = \{x \in \mathbb{R}| x \neq 0\}$ or in Interval-Notation: $(-\infty,0) \cup (0, \infty)$\\ 
	\item $R(f) = \{ f \in \mathbb{R} | f > 0\}$ or in Interval-Notation: \\$(0,\infty)$\\
\end{itemize}	
	
	
\end{enumerate}


\item (\textbf{30 points}) Find the limits. (Hint: try to simplify as much as possible before applying the limit!)

\begin{enumerate}
	\item $\lim_{x \to 0} \frac{3(x-1)+3}{x}$\\
	
	\textbf{Solution:}\\
	
\begin{align*}
	\lim_{x \to 0} \frac{3(x-1)+3}{x} &= \lim_{x \to 0} \frac{3x-3+3}{x}\notag\\
	&= \lim_{x \to 0} \frac{3x}{x}\notag\\
	&= 3\notag
\end{align*}	
	
	\item $\lim_{x \to 2} \frac{x-2}{x^2+x-6}$\\
	
	\textbf{Solution:}\\
	
\begin{align*}
	\lim_{x \to 2} \frac{x-2}{x^2+x-6} &= \lim_{x \to 2} \frac{x-2}{(x-2)(x+3)}\notag\\
	&= \lim_{x \to 2} \frac{1}{x+3}\notag\\
	&= \frac{1}{5}\notag
\end{align*}		
	
	
	\item $\lim_{x \to 1} \frac{x^2-4x+3}{x^2+x-2}$\\
	
	\textbf{Solution:}\\
	
\begin{align*}
	\lim_{x \to 1} \frac{x^2-4x+3}{x^2+x-2} &= \lim_{x \to 1} \frac{(x-1)(x-3)}{(x-1)(x+2)}\notag\\
	&= \lim_{x \to 1} \frac{x-3}{x+2}\notag\\
	&= -\frac{2}{3}\notag
\end{align*}	
	
\end{enumerate}


\end{enumerate}

\end{document}
